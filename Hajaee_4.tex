\documentclass[11pt,oneside,a4paper]{article}
\usepackage{url}
\usepackage{hyperref}
\usepackage{lmodern}
\usepackage[pdftex]{graphicx}
\usepackage{textcomp}
\usepackage[utf8]{inputenc}
\usepackage[finnish]{babel}
\usepackage[T1]{fontenc}
\title{D- ja Erlang-kielten datan kapselointi}
\author{Hansi Keijonen, Jari Koskinen, Eero Laine}

\begin{document}

\maketitle

\newpage

\section{Jarin osuus}
\subsection{Rakenteiset tyypit}
Alkeistyypeistä voidaan muodostaa taulukoita, joita on kahdenlaisia. Yksi
taulukoista on yleinen perustapaus array, jollainen löytyy myös C/C++-kielistä
\cite{KRR88}. Taulukon alkioihin voidaan viitata indeksin avulla.
\begin{verbatim}
int[] lukuja;
lukuja[11] = 13;
\end{verbatim}

Toinen taulukkomuoto on avaimen ja arvon sisältävä pari, associative array.
Taulukon alkio sisältää arvon ja siihen viittaava indeksi voi olla esimerkiksi
merkkijono: 
\begin{verbatim}
  int[string] kuukausi;
  
  kuukausi["tammikuu"] = 1;
  kuukausi["helmikuu"] = 2;
  
  writeln("Tammikuu=", kuukausi["tammikuu"]);
\end{verbatim}
Koodi tulostaa:
\begin{verbatim}
Tammikuu=1
\end{verbatim}

Lisäksi taulukko voidaan jättää dynaamiseksi, määrittämällä sen pituudeksi [],
jolloin sille voidaan osoittaa jokin olemassa oleva taulukko myöhemmin koodissa.

Alkeistyypeistä voidaan myös muodostaa tietueet struct tai union. Molemmat
vastaavat C/C++-kielen vastaavia rakenteita \cite{KRR88} ja ovat arvotyyppejä
\cite{DLA13}. Struct määritellään, ja sitä käytetään seuraavasti:

\begin{verbatim}
struct palkansaaja{
  int palkka;
  string titteli;
}

void main() {
  palkansaaja[5] palkolliset;
  palkolliset[1].palkka = 4500;
  palkolliset[1].titteli = "ohjelmoija";

  write(palkolliset[1].titteli, " tienaa ");
  writeln(palkolliset[1].palkka, " kuukaudessa");
}
\end{verbatim}

Union eroaa structista siten, että muuttujien arvot on talletettu muistissa
samaan kohtaan \cite{ALE10};\cite{KRR88}, riippumatta niiden tyypistä ja
pituudesta. Kääntäjän tehtävä on varata riittävä määrä muistia suurimman tyypin
mukaisesti. Yhden muuttujan arvon muuttaminen vaihtaa muuttujille varatun
muistin sisällön. Seuraava esimerkki havainnollistaa tätä:
\begin{verbatim}
union moniTyyppi{
  int iluku;
  uint uiluku;
  ubyte ubluku;
}

void main() {
  moniTyyppi luku;
  luku.iluku = 6200;
  writeln("int: ", luku.iluku);
  writeln("uint: ", luku.uiluku);
  writeln("ubyte: ", luku.ubluku);
  luku.iluku = -6200;
  writeln("int: ", luku.iluku);
  writeln("uint: ", luku.uiluku);
  writeln("ubyte: ", luku.ubluku);
}
\end{verbatim}
Ohjelma tulostaa seuraavasti:

\begin{verbatim}
int: 6200
uint: 6200
byte: 56
int: -6200
uint: 4294961096
ubyte: 200
\end{verbatim}

Lisäksi voidaan luoda dymaaninen taulukko, joka tarkoittaa sitä, että taulukkoon
voidaan myöhemmin sijoittaa taulukko. Esimerkki havainnollistaa tätä toimintaa:

\begin{verbatim}
  int[] c;
  int[4] d;
  d[2] = 10;
  c = d;
  writeln(c[2]);
\end{verbatim}
Ohjelma tulostaa luvun 10. \\

D-kielen tarjoama struct ja union ovat useissa tilanteissa käyttökelpoisia,
varsinkin matalammalla tasolla, kuten esimerkiksi käyttöjärjestelmien
toteutuksessa. Koska ne ovat arvotyyppejä, niillä saadaan aikaan tehokkaita
rakenteita. D-kielen kehitykseen vaikuttaneista kielistä Java ei sisällä struct,
eikä union rakennetta. C\# sitä vastoin sisältää struct arvotyypin rakenteen
mutta ei unionia.

\subsection{Geneerisyys}
D-kielessä on tuki geneerisyydelle. Tämä tarkoittaa sitä, että funktio voidaan
kirjoittaa yleiseksi ilman, että sen parametreja sidotaan tiettyihin tyyppeihin.
Tällöin parametrin tyypiksi määritetään T. Seuraava esimerkki on toteutettu
geneerisyyttä hyödyntäen; binäärihaku, jolle voidaan antaa syötteeksi minkä
tahansa tyyppinen järjestetty taulukko.

\begin{verbatim}
bool binHaku(T)(T[] input, T value) {
  while (!input.empty) {
    int i = input.length / 2;
    auto mid = input[i];
    if (mid > value)
      input = input[0 .. i];
    else 
      if (mid < value)
        input = input[i + 1 .. $];
    else 
      return true;
  }
  return false;
}

void main() {
  writeln(binHaku([ 1, 3, 6, 7, 9, 15 ], 6));
  writeln(binHaku([ 'a', 'b', 'c', 'd', 'e', 'g', 'i' ], 'h'));
}
\end{verbatim}

Funktio hyväksyy syötteeksi järjestetyn taulukon ja suorittaa puolitushaun
sille. Ohjelman tuloste on seuraava:
\begin{verbatim}
true
false
\end{verbatim}

\subsection{Mixin}
D-kieli tarjoaa mixinin. Mixin on kuin geneerinen luokka tai template. Ero
templaten ja mixinin välillä on se, että template alustetaan siihen
näkyvyysalueeseen, missä  alustus tehdään kun taas mixin voidaan alustaa mihin
tahansa näkyvyysalueeseen, kuten esimerkikis structiin [lähde?]. Alla on
määritetty template Kirjoittaja, joka kirjoittaa metodille kirjoita annetun
tiedon. 
\begin{verbatim}
template Kirjoittaja(T)
{
    void kirjoita(T t)
    {
        writefln(t);
    }
}
\end{verbatim}
Tällainen template voidaan instantioida paikalliseen näkyvyysalueeseen (ja
kutsua myös metodia kirjoita) seuraavalla tavalla:
\begin{verbatim}
Kirjoittaja!(int).kirjoita(666); 
\end{verbatim}
Template Kirjoittaja voidaan alustaa mixinin avulla myös struct:in sisälle:
\begin{verbatim}
struct S
{
    mixin Kirjoittaja!(int) IntKirjoittaja;
    mixin Kirjoittaja!(char[]) StrKirjoittaja;
} 
\end{verbatim}

\subsection{Luokat ja periytyminen}
D-kielessä luokka voi periytyä vain yhdestä luokasta, toisin kuin C++ -kielessä,
jossa moniperintä on mahdollinen. D:ssä aliluokka perii kaikki yliluokan
tietueet ja funktiot. 
\begin{verbatim}
class Henkilo {
  string nimi;
  int ika;
  // luokan konstruktori
  this(string nimi, int ika) {
    this.nimi = nimi;
    this.ika = ika;
  }
  ~this() {} // tyhjäksi jätetty destruktori
}

class Opiskelija : Henkilo { // perii luoka Henkilo ominaisuudet
  string opiskelijaNumero;
  int opintoPisteet;
  string opintoLinja;
  // luokan konstruktori
  this(string nimi, int ika, string opiskelijaNumero) { 
    super(nimi, ika); // kutsuu yliluokan konstruktoria
    this.opiskelijaNumero = opiskelijaNumero;
  }
}

void main() {
  Opiskelija kapistelija = new Opiskelija("Kerttu Koodari", 29, "987234651");
  Opiskelija konnari = new Opiskelija("Kalle Konnari", 34, "132435467");
  kapistelija.opintoLinja = "Tietojenkasittelytiede";
  konnari.opintoLinja = "Kognitiotiede";
  kapistelija.opintoPisteet = 123;
  konnari.opintoPisteet = 87;
  writeln(kapistelija.nimi, "n opintopistekertyma on ",
kapistelija.opintoPisteet);
  writeln(konnari.nimi, "n opintopistekertyma on ", konnari.opintoPisteet);
}
\end{verbatim}
Ohjelma tulostaa seuraavasti:

\begin{verbatim}
Kerttu Koodarin opintopistekertyma on 123
Kalle Konnarin opintopistekertyma on 87
\end{verbatim}
Perinnän voi estää kirjoittamalla luokkamäärittelyn eteen final \cite{DLA13}.
Rajapintaluokka löytyy myös ja se määritetään luokkamäärittelyn edessä
avainsanalla interface. 
Abstrakti luokka voidaan muodostaa avainsanalla abstract, joka kirjoitetaan
luokan määrityksen eteen. Abstrakti luokka voi sisältää D-kielessä abstrakteja
funktioita, joille aliluokan on annettava toteutus, ja normaaleja funktioita.
Abstraktista luokasta ei voi luoda ilmentymää, vaan ainoastaan abstraktin luokan
aliluokasta voidaan luoda ilmentymä. Esimerkkikoodia abstraktin luokan
toteutuksesta:
\begin{verbatim}
// abstrakti luokka Tervehdys
abstract class Tervehdys { 
  void tervehdi(){ 
    writeln("Hei!"); 
  } 
  abstract void tervehdiNimella(string name); 
} 

class TervehdysNimella : Tervehdys { 
  // tervehdiNimella abstraktin funktion toteutus
  void tervehdiNimella(string nimi){ 
    writeln("Hei ", nimi, "!"); 
  } 
} 

void main() {

  Tervehdys tervehdys = new TervehdysNimella(); 
  tervehdys.tervehdi(); 
  tervehdys.tervehdiNimella("Kerttu"); 
}
\end{verbatim}
Ja ohjelma tulostaa:
\begin{verbatim}
Hei!
Hei Kerttu!
\end{verbatim}

D-kielessä on lisäksi tuki rajapintaluokille, joita voidaan periä useampia
yhdelle luokalle. Rajapintaluokassa määritellään funktiot, jotka aliluokan
täytyy toteuttaa. Alla esimerkki koodista, jossa luokassa Laskenta toteutetaan
rajapintaluokkien Summa ja Tulo funktiot.

\begin{verbatim}
interface Summa { 
  int summa(int a, int b); 
} 

interface Tulo {
  int tulo(int a, int b);
}

class Laskenta : Summa, Tulo { 
  int summa(int a, int b) { 
    return a+b;
  } 

  int tulo(int a, int b) {
    return a*b;
  }
} 

void main() {
  Laskenta laskuri = new Laskenta();
  writeln("3+4=", laskuri.summa(3, 4));
  writeln("3*4=", laskuri.tulo(3, 4));
}
\end{verbatim}
Esimerkkikoodi tulostaa:
\begin{verbatim}
3+4=7
3*4=12
\end{verbatim}
\subsection{Arvo- ja viitesemantiikka}
D-kielessä tietueet, struct ja union, noudattavat arvosemantiikkaa. Luokat
noudattavat viitesemantiikkaa. Arvosemantiikkaa noudattavat tietueet
tallennetaan muistissa pinoon ja niiden olemassaolo riippuu näkyvyysalueesta.
Viitesemantiikkaan perustuvien luokkien ilmentymille taas varataan muistia
keosta; pinoon laitetaan vain osoitin keon kohtaan, jossa luotu olio sijaitsee.
Funktiokutsun parametrit voidaan välittää arvoina tai viitteinä, samaan tapaan
kuin C/C++ -kielissä \cite{KRR88}. Kielessä on tätä varten varattu merkit * ja
\&, josta lyhyt esimerkki:
\begin{verbatim}
void main() {
  int x=10;
  int y=20;
  writeln(x,",",y);
  vaihda(&x, &y);
  writeln(x,",",y);
}

void vaihda(int *px, int *py)
{
  int temp;			
  temp=*px;
  *px=*py;
  *py=temp;
}
\end{verbatim}
Koodissa vaihdetaan muistissa x:n ja y:n arvoja keskenään. main kutsuu funktiota
vaihda ja funktiolle välitetään parametrina x:n ja y:n muistiosoitteet \&x ja
\&y. vaihda-funktion parametrien tyypeiksi on määritetty osoittimet merkillä *.
Funktio tekee arvojen vaihtamisen suoria muistiosoitteita käyttäen. Tuloste on
seuraava:
\begin{verbatim}
10,20
20,10
\end{verbatim}

\section{Hansin osuus}
Erlangissa ei ole luokkia, joten ei ole myöskään mitään luokkiin liittyvää
toiminnallisuutta kuten perintää. Myöskään geneerisiä tyyppejä tai ajonaikaista
tyyppiparametrointia ei ole. 
Parametrinvälityksessä ja muuttujiin sijoittamisessa Erlangissa on aina käytössä
arvosemantiikka. 

Erlangin tietotakenne Record on hyvin samantapainen kuin c:n struct
\cite{HEB13}. Se on sopiva pienen tietorakenteen luomiseksi. Record määritellään
moduulin atribuutiksi:
 \begin{verbatim}
 -module(piste) .
 -compile(export_all) .
 -record( piste, {
    x,
    y
 }) .
 \end{verbatim}
 Esimerkissä on yksinkertainen tietorakenne pisteen kuvaamiseksi. Recordin
alustus samaisessa moduulissa tapahtuu seuraavasti: 
\begin{verbatim}
 piste() -> 
     #piste{
         x = 20,
         y = 35
     } .
 \end{verbatim}
Moduulin käännöksen jälkeen voidaan tulostaa pisteen kaikki tiedot tai aioastaan
yksittäinen tieto:
\begin{verbatim}
 >c(piste) .
 {ok, piste}
 >piste:piste() .
 #piste{x = 20, y = 35}
 >piste#.piste.x .
 20
 \end{verbatim}


 Tärkeä joukko Erlangin tietorakenteita on avain-arvo-parit (key-value stores).
Yleisin näistä on proplist, joka on tuplelista muotoa [\{key,value\}]. Muita
rajoitteita ei juuri ole. Proplistin käsittelyyn (lisäys, poisto, haku jne.)
löytyy moduulista proplists kaikki tarvittavat funkiot \cite{HEB13}. 
 
 Hieman formaalimmin määritelty avain-arvo -tietorakene on orddict eli
järjestetty sanakirja (ordered dctionary). Siinä avain saa esiintyä ainoastaan
kerran ja rakenne tarjoaa rajoitetun CRUD-toiminnallisuuden elementtien
tallentamiseen, etsimiseen, lukemiseen ja poistamiseen. Elementit ovat
järjestetty, joten haut ovat nopeita\cite{HEB13}. orddict on tehokas 75
elementin säilömiseen saakka. Tätä suuremmat tietomäärät kannattaa tallettaa
esimerkiksi dict:iin tai gb\textunderscore tree:hin. 
 
 dict:ien toiminnallisuus on lähes sama kuin orddict:issä  lisättynä muutamilla
funktioilla kuten fold ja map, jotka helpottavat tiedon käsittelyä. Molemmat funktiot
ovat tarkoitettu listan läpikäyntiin siten, että listan jokainen alkio käsitellään parametrina annetun funktion määräämällä tavalla. 	
 
 Erlang tarjoaa myös valmiin puurakenteen, gb\textunderscore treen \cite{HEB13}.
gb\textunderscore tree on tasapainotettu puu, johon on valmiiksi toteutettu
yleisimmät puissa tarvittavat funktiot kuten insert, delete, lookup muutamia
mainitakseni. Puurakenne on varsin tehokas pl. tilanteet, joissa tasapainotusta
joudutaan tekemään. 
 
 Eräs Erlangin erikoisuus on valmiiksi toteutettu suunnattu verkko, digraph
(directed graph) \cite{HEB13}. digraph on toteutettu kahdessa moduulissa digraph
ja digraph\textunderscore utils, joista edellinen toteuttaa verkon ja
jälkimmäinen tarjoaa palvelut verkon läpikäyntiin, renkaiden (cycle) löytämisen
jne. 
 
 Erlangissa on myös valmis toteutus FIFO-jonolle, nimeltää queue \cite{HEB13}.
Luonnollisestikin jonon toteutus sisältää funktiot elementtien lisäämiseksi
jonoon ja poistamiseksi jonosta. 
 
 Melko triviaali tietorakenne on taulukko (array), johon voi tallentaa
ainoastaan numeerisia alkioita \cite{HEB13}. Erlangin taulukko voidaan alustaa dynaamiseksi, jolloin sen koko kasvaa tarvittaessa
tai staattiseksi, jolloin sen koko pysyy samana koko olemassaolon ajan. Taulukko ei ole tehokas tietorakenne. Yleinen
käytäntö esimerkiksi raskaissa matriisioperatioissa on teettää työ muilla
kielillä kirjoitetuilla ohjelmilla käyttäen Erlangin siihen tajoamaa tekniikkaa
port:ia	
 

\section{Eeron osuus}

Erlang-kielessä set-tietorakenteet eli joukot ovat elementtien kokoelmia
(collection), joissa mistään elementistä ei ole kaksoiskappaleita. Erlangissa on
neljä moduulia joukkojen käsittelyyn: ordsets, sets, gb\_sets ja sofs (sets of
sets). Fred Hébertin mukaan suunnittelijoiden tausta-ajatuksena oli, ettei
joukon esittämiseen ole yhtä, optimaalista tapaa. (hepe)

ordsets-moduulissa järjestyksessä olevaa listaa käytetään joukon elementtien
tallentamiseen. (manuaali) ordsets-rakenteet ovat hyödyllisiä lähinnä pienten
joukkojen esittämiseen. ordsets-rakenteet ovat hitaita, mutta niiden esitystapa
on kaikista Erlangin joukkorakenteista yksinkertaisin ja helppolukuisin. Alla on
esimerkkejä ordsets-moduulin funktioista. (hepe)

\begin{verbatim}
ordsets:new/0
ordsets:is_element/2
ordsets:add_element/2
ordsets:del_element/2
ordsets:union/1
ordsets:intersection/1
\end{verbatim}

ordsets-moduulin esitystapa eroaa sets-moduulin esitystavasta yhdellä tavalla.
Siinä missä sets-moduuli olettaa kahden elementin olevan erilaisia, jos ne eivät
ole täysin samat (=:=), ordsets-moduulissa kaksi elementtiä eroavat toisistaan
ainoastaan, jos ne eivät ole yhtä suuret (==). (manuaali)

sets-moduuli on toteutettu käyttäen samankaltaista rakennetta kuin
dict-tietorakenne, ja se toteuttaa saman rajapinnan kuin ordsets-moduuli.
sets on kuitenkin paremmin skaalautuva, eli suurien elementtimäärien käsittely
on sets-moduulin kautta toteutetuissa joukoissa tehokkaampaa. Dict-rakenteiden
tapaan sets-moduulilla toteutetut joukot ovat erityisen hyviä
lukemispainotteisissa operaatioissa. Tällaisessa voitaisiin esimerkiksi
tarkastaa, onko jokin elementti mukana joukossa vai ei. (hepe)

gb\_sets-moduulissa järjestetyt joukot on toteutettu soveltaen General Balanced
Trees -konseptia. General Balanced Tree on yksinkertaisesti binääripuu, jolla on
kyky korjata muotonsa tarvittaessa. (andersson) Joukkojen toteuttaminen
gb\_sets-moduulin avulla voi olla suurien
joukkojen kohdalla paljon tehokkaampaa kuin järjestettyjen listojen käyttö.
(manuaali)

gb\_sets-moduuli on tehokas muissa operaatioissa kuin lukemisessa. Vaikka
gb\_sets toteuttaa saman rajapinnan kuin sets ja ordsets, se tarjoaa enemmän
funktioita.
Esimerkkeinä mainittakoon älykkäät ja naivit funktiot, iteraattorit ja nopea
pienimpien ja suurimpien arvojen haku.(hepe)

sofs-moduuli toteutetaan järjestetyillä listoilla, jotka on suljettu metadatan
kanssa tuple-tyypin sisälle. sofs-moduuli on hyödyllinen, jos ohjelmoija haluaa
täyden hallinnan joukkojen ja perheiden (??) välisistä suhteista tai esimerkiksi
pakottaa (?) joukkojen tyyppejä. sofs-moduulilla toteutetut joukot ovat ikään
kuin lähellä matemaattisen joukon konseptia, ne eivät ole ainoastaan
ainutkertaisten elementtien ryhmiä.(hepe)

%The sofs module implements operations on finite sets and relations represented
%as sets. Intuitively, a set is a collection of elements; every element belongs
% %to the set, and the set contains every element.

Erlangin kehittäjäryhmän jäsen Björn Gustavsson suosittelee gb\_sets-moduulin
käyttöä useimmissa tilanteissa. Gustavsson käyttäisi ordsets-moduulia silloin,
kun ohjelmoija haluaa selkeän esityksen siitä, mitä haluaa koodissaan
prosessoitavan. sets-moduulia Gustavsson suosittelee käytettäväksi ainostaan
silloin, kun operaattoria =:= tarvitaan.  (hepe)

\bibliography{Lähteet}

  
\begin{thebibliography}{99}	

\bibitem[ERL99]{ERL99} Erlang 4.7.3 Reference Manual, DRAFT (0.7), Jonas
Barklund, Robert Virding, 1999. 

\bibitem[DLA13]{DLA13} http://dlang.org, noudettu 6.2.2013.

\bibitem[HEB13]{HEB13} Learn You Some Erlang For Great Good, Fred Hébert, 2013.

\bibitem[AND13]{AND13} http://user.it.uu.se/~arnea/ps/gbimpl.pdf, noudettu
20.2.2013

\bibitem[ERL13]{ERL13} http://www.erlang.org/doc, noudettu 20.2.2013

\bibitem[ALE10]{ALE10} The D Programming Language, Andrei Alexandrescu, 2010.

\bibitem[KRR88]{KRR88} The C Programming Language, Brian W. Kernighan, Dennis M.
Ritchie, 1978/1988.


\end{thebibliography}

\end{document}