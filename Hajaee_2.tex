\documentclass[11pt,oneside,a4paper]{article}
%\usepackage{ucs}
\usepackage{url}
\usepackage{hyperref}
\usepackage{lmodern}
\usepackage[pdftex]{graphicx}
\usepackage{textcomp}
\usepackage[utf8]{inputenc}
\usepackage[finnish]{babel}
\usepackage[T1]{fontenc}
%opening
\title{D:n ja Erlangin tunnukset sekä kontrollirakenteet}
\author{Hansi Keijonen, Jari Koskinen, Eero Laine}

\begin{document}

\maketitle



\section{Tunnusten näkyvyysalueet ja sidonta}
(D ja Erlang vierekkäin)

Jo vuonna 1983 Hansi Keijonen todisti, että LaTeX-dokumenttien kirjoittaminen on työlästä, mutta palkitsevaa \cite{KEI83}. \\

D-kielessä lohkot rajataan merkein \{ \}, ja ne voivat koostua lausekkeista \cite{DLA13}. Lohkot sidotaan lausekkeeseen, luokkaan tai funktioon. Lohkolla voidaan ilmaista myös nimetön näkyvyysalue, jos sitä ei ole sidottu funktioon, luokkaan tai lausekkeeseen. Lohkon sisältä on näkyvyys lohkon ulkopuolelle, mutta ulkopuolelta ei ole näkyvyyttä lohkon sisälle. Luokkien ja niiden tietueiden näkyvyys voidaan kuitenkin sallia ulkopuolelle \cite{DLA13}.

lohkorakenne, litteä vs syvä \\
D-kielessä lohkot rajataan merkein \{ \}, ja ne voivat koostua lausekkeista.
Lohkot sidotaan lausekkeeseen, luokkaan tai funktioon. Lohkolla voidaan ilmaista
myös nimetön näkyvyysalue, jos sitä ei ole sidottu funktioon, luokkaan tai
lausekkeeseen. Lohkon sisältä on näkyvyys lohkon ulkopuolelle, mutta
ulkopuolelta ei ole näkyvyyttä lohkon sisälle. Luokkien ja niiden tietueiden
näkyvyys voidaan kuitenkin sallia ulkopuolelle.

Erlangin lähestymistapa on erilainen kuin D-kielessä. Lohkorakenne perustuu
produktioihin, jotka sisältävät otsikon ja rungon. Runko voi sisältää
lausekkeita ja yksi produktio voi sisältää useampia runkoja.

D-kielessä tunnusten sidonta tehdään staattisesti käännösaikana. Lohkon sisällä
ei voida suoraan esitellä uudelleen ulompien tasojen näkyviä muuttujia, mutta
funktioiden ja tietuiden sisällä tämä on sallittua.

 lohkorakenne voi olla hyvinkin syvä, se sallii funktion sijoittamisen toisen
funktion sisälle. Samoja muuttujia on mahdollista esitellä sisemmissä
funktioissa uudelleen. Muuttujan voi määritellä kuitenkin vain sisemmässä
funktiossa, esimerkiksi sisäkkäiset for-loopit vaativat kukin omat muuttujansa.
Esimerkki tällaisesta funktioiden sisäkkäin sijoittelusta, jossa muuttuja
määritetään uudelleen:

Jo vuonna 1983 Hansi Keijonen todisti, että LaTeX-dokumenttien kirjoittaminen on työlästä, mutta palkitsevaa \cite{KEI83}.
D-kielessä lohkot rajataan merkein \{ \}, ja ne voivat koostua lausekkeista. Lohkot sidotaan lausekkeeseen, luokkaan tai funktioon. Lohkolla voidaan ilmaista myös nimetön näkyvyysalue, jos sitä ei ole sidottu funktioon, luokkaan tai lausekkeeseen. Lohkon sisältä on näkyvyys lohkon ulkopuolelle, mutta ulkopuolelta ei ole näkyvyyttä lohkon sisälle. Luokkien ja niiden tietueiden näkyvyys voidaan kuitenkin sallia ulkopuolelle.

Erlangin lähestymistapa on erilainen kuin D-kielessä. Lohkorakenne perustuu produktioihin, jotka sisältävät otsikon ja rungon. Runko voi sisältää lausekkeita ja yksi produktio voi sisältää useampia runkoja. 

D-kielessä tunnusten sidonta tehdään staattisesti käännösaikana \cite{DLA13}. Lohkon sisällä ei voida suoraan esitellä uudelleen ulompien tasojen näkyviä muuttujia, mutta funktioiden ja tietuiden sisällä tämä on sallittua \cite{DLA13}.

Lohkorakenne voi olla D-kielessä hyvinkin syvä, se sallii funktion sijoittamisen toisen funktion sisälle. Samoja muuttujia on mahdollista esitellä sisemmissä funktioissa uudelleen. Muuttujan voi määritellä kuitenkin vain sisemmässä funktiossa, esimerkiksi sisäkkäiset for-loopit vaativat kukin omat muuttujansa. Esimerkki tällaisesta funktioiden sisäkkäin sijoittelusta, jossa muuttuja määritetään uudelleen:

 lohkorakenne voi olla hyvinkin syvä, se sallii funktion sijoittamisen toisen funktion sisälle. Samoja muuttujia on mahdollista esitellä sisemmissä funktioissa uudelleen. Muuttujan voi määritellä kuitenkin vain sisemmässä funktiossa, esimerkiksi sisäkkäiset for-loopit vaativat kukin omat muuttujansa. Esimerkki tällaisesta funktioiden sisäkkäin sijoittelusta, jossa muuttuja määritetään uudelleen:

\begin{verbatim}
int i=0;

int uloin() {
    for(int i=0; i<5; i++) {
        int sisin() {
            int j=i;
                for(int i=j;i<5; i++) {
                    writeln("s: ",i);
                }
            return i;
            }
        writeln("u: ", i);
        sisin();
    }
return i;
}
\end{verbatim}

Erlang on litteä kieli, siinä voi olla lohkoja vain yhdellä tasolla.
Erlang-kielen määrittely ei varsinaisesti esittele lohkosääntöjä. \\

D-kielessä sidonta tehdään staattisesti käännösaikana. Erlangissa sidonta on dynaaminen ja muuttujalle voidaan asettaa arvo vain kerran. D-kielessä muuttujan tilaa voidaan muuttaa suorituksen aikana.

Kaikki D-kielen muuttujat kuuluvat ns. ensimmäiseen luokkaan \cite{DLA13}. Muuttujia voidaan välittää parametrina, niihin voidaan sijoittaa arvoja, ja palauttaa funktion arvona. Kieli sallii myös funktioliteraalien sijoittamisen funktiokutsun parametriksi.

\begin{verbatim}
void loop(int k, int j, void delegate() statement) { 
  for (int i = k; i < j; i++) { 
    statement(); 
  }
  
loop(5, 100, { d += 1; } ); 
loop(3, 10, { f += 3; } );
\end{verbatim}
=======
sidonta, staattinen vs dynaaminen \\
D-kielessä sidonta tehdään staattisesti käännösaikana. Erlangissa sidonta on
dynaaminen.

ensimmäisen toisen ja kolmannen luokan arvot

sulkeumat

esimerkit

em. vertailu

\section{Kontrollin ja laskennan ohjaus - Erlang}
Jotta voidaan ymmärtää Erlangin kontrollirakenteita, pitää ensimmäisenä
selvittää hahmontunnistuksen mekanismi. Hahmontunnistus perustuu siihen, että
funktioon määritellään useita funktiokutsuja siten, että jokaisessa kutsussa
parametrit ovat lukumäärältään samat, mutta ovat arvoiltaan erilaiset. Ts. kun 
funktiota kutsutaan tietyllä parametrilla, käydään järjestyksessä läpi eri
funktion kutsuvaihtoehdot ja kun ensimmäinen parametreihin sopiva
kutsuvaihtoehto löytyy, 
toimitaan sen määrityksen mukaan. Helpoimmin asia selviää esimerkin avulla: 
\begin{verbatim}
1: tervehdi(mies, Nimi) -> io:format("Terppa, herra %s!",[Nimi]);
2: tervehdi(nainen,Nimi) -> io:format("Terppa, rouva %s!",[Nimi]);
3: tervehdi(_,Nimi) -> io:format("Terppa %s!",[Nimi]) . 
\end{verbatim}


Kutsutaessa tervehdi-funktiota tervehdi(nainen,"Tuppurainen"), tutkitaan
ensiksi, sopiiko parametrit  rivin 1 parametrisointivaihtoehtoon. Koska
parametrit olivat epäsopivaiset, 
yritetään kutsua sovittaa rivin kaksi parametrisointivaihtoehtoon. Tällä kertaa
parametrit sopivat ja ohjelma tulostaa "Terppa, rouva Tuppurainen!". Jos
funktiota olisi kutsuttu
tervehdi(hermafrodiitti,"Höttölä"), olisi vasta rivin 3 parametrit osuneet
oikeaan, ja silloin olisi toimittu tämän rivin määritysten mukaan. 
Rivin 3 toinen parametri on universaalimerkki, 
johon voi sovittaa minkä tahansa arvon.

Valintojen tekemiseen Erlangissa on em. hahmonsovituksen lisäksi mahdollista
käyttää if-lauseita, guardeja sekä in case of -rakennetta.

Erlangissa if-rakenne ilmaistaan hieman esim. Javasta ja D:stä poikkeavalla
tavalla, jota valotan seuraavassa esimerkissä. 

\begin{verbatim}
1: vertaa(N,M) -> 
2: 		if N>M -> 1;
3:	     N<M -> -1;
4:	     true -> 0
5: 	 end . 
\end{verbatim}

Esimerkkifunktiossa siis verrataan kahta lukua ja toimitaan eri tavoin sen
mukaan, mikä on vertailun tulos. Riveillä 2 ja 3 tutkitaan ovatko N ja M
erisuuruiset, mutta rivillä 4 on hieman erikoiselta
vaikuttava true, joka vastaa esim. D:n else:ä. Koko if-rakenne päätetään end
sanaan ja pisteeseen. 

Guardit ovat yksi Erlangin erikoisuus, joka on tuttu myös Haskellista. Guardit
ovat keino tehdä hahmonsovituksesta hieman ilmaisuvoimaisempaa. Sen sijaan, että
täysi-ikäisyyden osoittava ohjelma 
toteutettaisiin näin: 
\begin{verbatim}
taysi_ika(1) -> false;
taysi_ika(2) -> false;
taysi_ika(3) -> false;
.
.
.
taysi_ika(16) -> false;
taysi_ika(17) -> false;
taysi_ika(_) -> true;
\end{verbatim}
, toteutetaan iän tarkastaminen guardilla
\begin{verbatim}
taysi_ika(N) when N < 18 -> false;
taysi_ika(_) -> true . 
\end{verbatim}

Tällä tavoin voidaan jättää kirjoittamatta useita funktiokutsuvaihtoehtoja. 
Tärkeää on muistaa, että guard-rakenteen tulee aina palauttaa true jollain
vaihtoehdolla, muuten kääntäjä ilmoittaa virheestä.  

Erlangissa on myös case…of -rakenne, joka mahdollistaa monipuoliset
valintarakenteet, johon voidan yhdistää myös guardeja. Esimerkki selventää
rakennetta helpoiten:
\begin{verbatim}
1: biitsi(Lampotila) -> 
2:    case Lampotila of
3:        {calsius, N} when N >= 20, N =< 45 -> 'ihan jees';
4:        {kelvin, N}  when N >= 293, N =< 318 -> 'tieteellisesti jees';  
5:        {fahrenheit, N}  when N >= 68, N =< 113 -> 'amerikkalaisittain jees'; 
6:         _ -> 'ei niin jees' 
\end{verbatim}

Ohjelma siis saa parametrinaan tuplen, jossa on lämpötila-asteikko ja astemäärä.
Sen jälkeen case…of -rakenteessa riveillä  tutkitaan, minkä periaatteen
mukaisesti astemäärää tutkitaan. Lopulta guardin avulla päätetään, onko
parametrina 
saatu lämpötila sopiva rantaelämään. Mielenkiintoinen kysymys on se, että mihin
case…of -rakennetta tarvitaan, koska se on hyvin samanlainen funktiokutsujen
hahmonsovituksen kanssa. Valinta kannattaa perustaa henkilökohtaisiin
mieltymyksiin. 

Erlangissa toistoa ja rekursiota ei voi erottaa toisistaan, koska rekursio on
ainoa tapa suorittaa toistorakenteita. Erlangissa ei ole siis for- tai while
-lauseita esim. D:n tai Javan tapaan. Listan läpikäynti Erlangissa perustuu
kielen ominaisuuteen, että
esim. koknaislukuja sisältävä lista [1,2,3,4]  voidaan ilmoittaa myös
konstruktorioperaattoria '|' käyttäen [1|[2,3,4]], eli luku 1 liitetään listaan
[2,3,4]. Lista voidaan ilmoittaa myös head- ja tail-funktioita käyttäen
[Head|Tail] = [1,2,3,4] jolloin Head = 1 ja Tail = [2,3,4].
Jos halutaan käydä lista läpi yksinkertaisesti siten, että laskemme jokaisen
listan alkion yhteen, voidaan se Erlangissa tehdä esim. seuraavalla tavalla:

\begin{verbatim}
1: summa([]) -> 0;
2: summa(Lista) -> head + summa(Tail) .  
\end{verbatim}


Tässä funktiossa siis hahmonsovituksen avulla toteutetaan rekursiivinen listan
läpikäynti. Rivillä 1 on toteutettu alkeistapaus, jossa lista on tyhjä ja
jolloin alkioiden summa on 0. Rivillä kaksi summataan listan ensimmäinen alkio
(Head) loppulistan 
(Tail) alkioiden summan kanssa. Tällä tavoin voidaaan tehdä monipuolisia listan
läpikäyntejä. 

Rekursio on Erlangissa yvin tavanomainen tapa tehdä asioita, semminkin kun muita
toistorakenteita ei ole. Perinteinen rekursiolla tehtävä funktio on kertoma n!,
joka voidaan ilmaista yhtälöparilla 

\begin{displaymath}
n! = \left\{ \begin{array}{ll}
1 & \textrm{jos n=0}\\
n*((n-1)!) & \textrm{{jos n>0}}
\end{array} \right.
\end{displaymath}

Tämä on on helposti ja ulkoasultaan elegantisti toteutettavissa Erlangilla
hahmonsovitusta ja rekursiota hyväksikäyttäen:
\begin{verbatim}
kertoma(0) -> 1;
kertoma(N) when N > 0 -> N*(kertoma(N-1)) .
\end{verbatim}
Jälleen aluksi tarkistetaan alkeistapaus, joka on tässä tapauksessa luvun 0
kertoma ja jos tämä rivi ei ota kiinni, suoritetaan seuraavan rivin
rekursiivinen kutsu, jossa parametrilla N kerrotaan rekursiivisesti kutsuttavan
kertoma(N-1) tulos.
Kun rekursiokutsut saavuttavat tapauksen kertoma(0), palautuu tulos pitkin
kutsuketjua alkuun. Tämän lähestymistavan huono puoli on se, että tietokoneen
muistissa joudutaan pitämään koko ajan jokaisen kutsun 'välitulos'. Esimerkiksi
edellisessä 
kertomaesimerkissä parametrilla 3 laskenta menisi seuraavasti:
\begin{verbatim}
kertoma(3) = 3*(kertoma(2))
                 = 3*(2*(kertoma(1))
                 = 3*(2*(1*(kertoma(0))))
                 = 3*(2*(1*1))
                 = 3*(2*1)
                 = 3*(2)
                 = 6
\end{verbatim}
Joutuisimme pitämään muistissa kaikki nuo 7 välivaihetta. Tässä tapauksessa tämä
ei olisi ongelma, mutta jos parametrina annettaisin joku todella suuri luku,
muistin riittävyys muodostuisi ongelmaksi. 
Tämän ongelman voi ohittaa häntärekursiolla. Häntärekursiivinen kertomafunktio
toteutetaan toteutamalla myös apufunktio hantakertoma(N,Acc) johon otetaan
mukaan vielä toinen parametri Acc, johon tallennetaan laskennan 
välivaiheiden tulokset:
\begin{verbatim}
kertoma(N) -> hantakertoma(N,1) .

hantakertoma (0,Acc) -> Acc;
hantakertoma(N,Acc) when N > 0 -> hantakertoma(N-1, Acc*N) .
\end{verbatim}

Nyt laskennan kulku on seuraavanlainen: 
\begin{verbatim}
kertoma(3) = hantakertoma (3,1)
                 = hantakertoma (3-1, 3*1)		
                 = hantakertoma (2-1,2*3)
                 = hantakertoma (1-1,1*6)
                 = 6
\end{verbatim}
Tässä tapauksessa joudumme pitämään ainoastaan kaksi lukua muistissa koko
laskennan ajan kun normaalissa rekursiossa joudumme pitämään kaikki
rekursiokutsuketjun välivaiheet muistissa. Jos unohdetaan se, kuinka paljon
luvun 3 ja 1000 000 esittäminen kuluttaa muistia, kertoma(3) ja kertoma(1000000)
vaativat saman vakiomäärän muistia. 

\section{Kontrollin ja laskennan ohjaus - D}

\subsection{Valinta}

If-lause on toteutettu D:ssä melko tyypillisellä tavalla:

\begin{verbatim}
if (a==b) {
   writeln("A ja B ovat samoja.");
}
else {
   writeln("A ja B eivät ole samoja.");
}
\end{verbatim}
Toisin kuin muilla kielillä, D:llä ei ole ``tyhjän lauseen'' (empty statement)
konstruktiota. Koodinpätkä

\begin{verbatim}
if (a == b);
\end{verbatim}
ei siis ole D:n hyväksymää. Sen sijaan tulisi kirjoittaa

\begin{verbatim}
if (a == b) {}
\end{verbatim}
Else-lause on aina sidottu lähimpään if-lauseeseen. Esimerkiksi:

\begin{verbatim}
if (a == b)
   if (b == c)
      writeln("B ja C ovat samoja.");
   else
      writeln("B ja C eivät ole samoja.");
\end{verbatim}
Useat laskeutuvat (cascading) if-else -lauseet toteutetaan kuten C-kielessä.

D:n static if -lause on C-kielen \#if-rakennetta muistuttava kääntämisajan
(compile-time) selektori. Tullessaan static if -lauseen kohdalle kääntäjä arvioi
ohjaavan lausekkeen. Jos lauseke ei ole tyhjä, vastaava koodi käännetään. Muussa
tapauksessa ainoastaan mahdollisen else-lause käännetään. 

Static if -lauseen testaama lauseke voi olla mikä vain if-testattava lauseke,
jota voidaan arvioida kääntämisen aikana. Mahdollisia lausekkeisiin kuuluvat
myös funktiokutsut.

Goto-lause toimii D:ssa allaolevalla tavalla:

\begin{verbatim}
goto <nimi>;
\end{verbatim}
D:ssa goto-lauseen käyttöön ei liity paljoa rajoituksia, mikä D:n pääkehittäjä
Andrei Alexandrescun mukaan tekee gotosta ongelmallisen työkalun. Goto voi
hypätä eteen- tai taaksepäin ja sisään ja ulos lauseista. Symbolin <nimi> tulee
kuitenkin olla näkyvissä funktiossa, missä gotoa kutsutaan. 

\subsection{Toisto}

While- ja do-while -lauseet ovat hyvin samankaltaisia, kuin vaikkapa
Java-kielessä. 

\begin{verbatim}
while (i => 0)
   --i;
\end{verbatim}
Kuten Javassa, koodin suoritus alkaa arvioimalla while-ehtoa. Jos lauseke ei ole
tyhjä, silmukan sisällä olevaa lause suoritetaan, ja silmukka alkaa uudestaan
while-ehdon arvionnilla. Do - while -lauseessa vastaava lauseke suoritetaan heti
kerran ennen while-ehdon tarkastelua. Allaolevassa esimerkissä näkyy myös D:ssä
pakollinen puolipiste lauseen lopussa.

\begin{verbatim}
do (--i) while (i => 0);
\end{verbatim}
For-lause on edelleen samankaltainen kuin Javassa.

\begin{verbatim}
for (int i = 0; i < 10; i++) 
   teeJotain(i);
\end{verbatim}
Listarakenteiden käsittelyyn foreach-lauseesta on olemassa allaoleva muoto:

\begin{verbatim}
foreach(kappale ; joukko) {
   teeJotain(kappale);
}
\end{verbatim}
Foreach-lause toimii siis kuten esimerkiksi Java-kielessä. Kappaleen viite
sidotaan vuorotellen jokaiseen joukko-listan elementtiin, ja teeJotain() -metodi
suoritetaan jokaisen elementin kohdalla. 

\subsection{Rekursio}

Rekursio toimii D:ssä kuten Java-kielessä. Allaolevassa esimerkissä lasketaan
kokonaisluvun kertoma rekursion avulla.

\begin{verbatim}
int kertoma(int n) {
  if(n<2) {
    return 1;
  }
  else
  {
    return n * kertoma(n-1);
  }
}
\end{verbatim}
Main-metodissa suoritetaan kertoma-metodi, jolle annetaan parametriksi
kokonaisluku n. Kertoma-metodi palauttaa luvun yksi, jos n on yksi, muuten
kutsutaan rekursiivisesti kertoma-metodia. Lukua siis kerrotaan itseään yhtä
pienemmän luvun kertomalla, kunnes päästään lukuun 1.  

Häntärekursiolla toteutettuna metodi näyttää tältä:
\begin{verbatim}
int kertoma(int n, int sum) {
  if(n==0) 
    return sum;
  else
    kertoma(n-1, sum * n);
}
\end{verbatim}
Nyt yhden kertoma-metodin suoritus loppuu toisen alkaessa, ja kertoma-metodi pitää kertoman tuloa mukanaan. Viimeiseksi suoritettu kertoma-metodi palauttaa loppusumman.

%\bibliographystyle{tktl}
\bibliography{Lähteet}

\begin{thebibliography}{99}

\bibitem[DLA13]{DLA13} http://dlang.org, noudettu 6.2.2013.

\bibitem[KEI83]{KEI83} Hansi. M. O. Keijonen, LaTeX-dokumenttien kirjoittaminen, 1983.
\end{thebibliography}

\end{document}
