\documentclass[11pt,oneside,a4paper]{article}
%\usepackage{ucs}
\usepackage{url}
\usepackage{hyperref}
\usepackage{lmodern}
\usepackage[pdftex]{graphicx}
\usepackage{textcomp}
\usepackage[utf8]{inputenc}
\usepackage[finnish]{babel}
\usepackage[T1]{fontenc}
%opening
\title{D:n ja Erlangin tunnukset sekä kontrollirakenteet}
\author{Hansi Keijonen, Jari Koskinen, Eero Laine}

\begin{document}

\maketitle



\section{Tunnusten näkyvyysalueet ja sidonta}
(D ja Erlang vierekkäin)

lohkorakenne, litteä vs syvä

sidonta, staattinen vs dynaaminen

ensimmäisen toisen ja kolmannen luokan arvot

sulkeumat

esimerkit

em. vertailu

\section{Kontrollin ja laskennan ohjaus}
Jotta voidaan ymmärtää Erlangin kontrollirakenteita, pitää ensimmäisenä selvittää hahmontunnistuksen mekanismi. Hahmontunnistus perustuu siihen, että
funktioon määritellään useita funktiokutsuja siten, että jokaisessa kutsussa parametrit ovat lukumäärältään samat, mutta ovat arvoiltaan erilaiset. Ts. kun 
funktiota kutsutaan tietyllä parametrilla, käydään järjestyksessä läpi eri funktion kutsuvaihtoehdot ja kun ensimmäinen parametreihin sopiva kutsuvaihtoehto löytyy, 
toimitaan sen määrityksen mukaan. Helpoimmin asia selviää esimerkin avulla: 
\begin{verbatim}
1: tervehdi(mies, Nimi) -> io:format("Terppa, herra %s!",[Nimi]);
2: tervehdi(nainen,Nimi) -> io:format("Terppa, rouva %s!",[Nimi]);
3: tervehdi(_,Nimi) -> io:format("Terppa %s!",[Nimi]) . 
\end{verbatim}


Kutsutaessa tervehdi-funktiota tervehdi(nainen,"Tuppurainen"), tutkitaan ensiksi, sopiiko parametrit  rivin 1 parametrisointivaihtoehtoon. Koska parametrit olivat epäsopivaiset, 
yritetään kutsua sovittaa rivin kaksi parametrisointivaihtoehtoon. Tällä kertaa parametrit sopivat ja ohjelma tulostaa "Terppa, rouva Tuppurainen!". Jos funktiota olisi kutsuttu
tervehdi(hermafrodiitti,"Höttölä"), olisi vasta rivin 3 parametrit osuneet oikeaan, ja silloin olisi toimittu tämän rivin määritysten mukaan. 
Rivin 3 toinen parametri on universaalimerkki, 
johon voi sovittaa minkä tahansa arvon.

Valintojen tekemiseen Erlangissa on em. hahmonsovituksen lisäksi mahdollista käyttää if-lauseita, guardeja sekä in case of -rakennetta.

Erlangissa if-rakenne ilmaistaan hieman esim. Javasta ja D:stä poikkeavalla tavalla, jota valotan seuraavassa esimerkissä. 

\begin{verbatim}
1: vertaa(N,M) -> 
2: 		if N>M -> 1;
3:	     N<M -> -1;
4:	     true -> 0
5: 	 end . 
\end{verbatim}

Esimerkkifunktiossa siis verrataan kahta lukua ja toimitaan eri tavoin sen mukaan, mikä on vertailun tulos. Riveillä 2 ja 3 tutkitaan ovatko N ja M erisuuruiset, mutta rivillä 4 on hieman erikoiselta
vaikuttava true, joka vastaa esim. D:n else:ä. Koko if-rakenne päätetään end sanaan ja pisteeseen. 

Guardit ovat yksi Erlangin erikoisuus, joka on tuttu myös Haskellista. Guardit ovat keino tehdä hahmonsovituksesta hieman ilmaisuvoimaisempaa. Sen sijaan, että täysi-ikäisyyden osoittava ohjelma 
toteutettaisiin näin: 
\begin{verbatim}
taysi_ika(1) -> false;
taysi_ika(2) -> false;
taysi_ika(3) -> false;
.
.
.
taysi_ika(16) -> false;
taysi_ika(17) -> false;
taysi_ika(_) -> true;
\end{verbatim}
, toteutetaan iän tarkastaminen guardilla
\begin{verbatim}
taysi_ika(N) when N < 18 -> false;
taysi_ika(_) -> true . 
\end{verbatim}

Tällä tavoin voidaan jättää kirjoittamatta useita funktiokutsuvaihtoehtoja.  Tärkeää on muistaa, että guard-rakenteen tulee aina palauttaa true jollain vaihtoehdolla, muuten kääntäjä ilmoittaa virheestä.  

Erlangissa on myös case…of -rakenne, joka mahdollistaa monipuoliset valintarakenteet, johon voidan yhdistää myös guardeja. Esimerkki selventää rakennetta helpoiten:
\begin{verbatim}
1: biitsi(Lampotila) -> 
2:    case Lampotila of
3:        {calsius, N} when N >= 20, N =< 45 -> 'ihan jees';
4:        {kelvin, N}  when N >= 293, N =< 318 -> 'tieteellisesti jees';  
5:        {fahrenheit, N}  when N >= 68, N =< 113 -> 'amerikkalaisittain jees'; 
6:         _ -> 'ei niin jees' 
\end{verbatim}

Ohjelma siis saa parametrinaan tuplen, jossa on lämpötila-asteikko ja astemäärä. Sen jälkeen case…of -rakenteessa riveillä  tutkitaan, minkä periaatteen mukaisesti astemäärää tutkitaan. Lopulta guardin avulla päätetään, onko parametrina 
saatu lämpötila sopiva rantaelämään. Mielenkiintoinen kysymys on se, että mihin case…of -rakennetta tarvitaan, koska se on hyvin samanlainen funktiokutsujen hahmonsovituksen kanssa. Valinta kannattaa perustaa henkilökohtaisiin mieltymyksiin. 

Erlangissa toistoa ja rekursiota ei voi erottaa toisistaan, koska rekursio on ainoa tapa suorittaa toistorakenteita. Erlangissa ei ole siis for- tai while -lauseita esim. D:n tai Javan tapaan. Listan läpikäynti Erlangissa perustuu kielen ominaisuuteen, että
esim. koknaislukuja sisältävä lista [1,2,3,4]  voidaan ilmoittaa myös konstruktorioperaattoria '|' käyttäen [1|[2,3,4]], eli luku 1 liitetään listaan [2,3,4]. Lista voidaan ilmoittaa myös head- ja tail-funktioita käyttäen [Head|Tail] = [1,2,3,4] jolloin Head = 1 ja Tail = [2,3,4].
Jos halutaan käydä lista läpi yksinkertaisesti siten, että laskemme jokaisen listan alkion yhteen, voidaan se Erlangissa tehdä esim. seuraavalla tavalla:

\begin{verbatim}
1: summa([]) -> 0;
2: summa(Lista) -> head + summa(Tail) .  
\end{verbatim}


Tässä funktiossa siis hahmonsovituksen avulla toteutetaan rekursiivinen listan läpikäynti. Rivillä 1 on toteutettu alkeistapaus, jossa lista on tyhjä ja jolloin alkioiden summa on 0. Rivillä kaksi summataan listan ensimmäinen alkio (Head) loppulistan 
(Tail) alkioiden summan kanssa. Tällä tavoin voidaaan tehdä monipuolisia listan läpikäyntejä. 

Rekursio on Erlangissa yvin tavanomainen tapa tehdä asioita, semminkin kun muita toistorakenteita ei ole. Perinteinen rekursiolla tehtävä funktio on kertoma n!, joka voidaan ilmaista yhtälöparilla 

kaava!!!!

Tämä on on helposti ja ulkoasultaan elegantisti toteutettavissa Erlangilla hahmonsovitusta ja rekursiota hyväksikäyttäen:
\begin{verbatim}
kertoma(0) -> 1;
kertoma(N) -> N*(kertoma(N-1)) .
\end{verbatim}
Jälleen aluksi tarkistetaan alkeistapaus, joka on tässä tapauksessa luvun 0 kertoma ja jos tämä rivi ei ota kiinni, suoritetaan seuraavan rivin rekursiivinen kutsu, jossa parametrilla N kerrotaan rekursiivisesti kutsuttavan kertoma(N-1) tulos.
Kun rekursiokutsut saavuttavat tapauksen kertoma(0), palautuu tulos pitkin kutsuketjua alkuun. Tämän lähestymistavan huono puoli on se, että tietokoneen muistissa joudutaan pitämään koko ajan jokaisen kutsun 'välitulos'. Esimerkiksi edellisessä 
kertomaesimerkissä parametrilla 3 laskenta menisi seuraavasti:
\begin{verbatim}
kertoma(3) = 3*(kertoma(2))
                 = 3*(2*(kertoma(1))
                 = 3*(2*(1*(kertoma(0))))
                 = 3*(2*(1*1))
                 = 3*(2*1)
                 = 3*(2)
                 = 6
\end{verbatim}
Joutuisimme pitämään muistissa kaikki nuo 7 välivaihetta. Tässä tapauksessa tämä ei olisi ongelma, mutta jos parametrina annettaisin joku todella suuri luku, muistin riittävyys muodostuisi ongelmaksi. 
Tämän ongelman voi ohittaa häntärekursiolla. Häntärekursiivinen kertomafunktio toteutetaan toteutamalla myös apufunktio hantakertoma(N,Acc) johon otetaan mukaan vielä toinen parametri Acc, johon tallennetaan laskennan 
välivaiheiden tulokset:
\begin{verbatim}
kertoma(N) -> hantakertoma(N,1) .

hantakertoma (0,Acc) -> Acc;
hantakertoma(N,Acc) when N > 0 -> hantakertoma(N-1, Acc*N) .
\end{verbatim}

Nyt laskennan kulku on seuraavanlainen: 
\begin{verbatim}
kertoma(3) = hantakertoma (3,1)
                 = hantakertoma (3-1, 3*1)		
                 = hantakertoma (2-1,2*3)
                 = hantakertoma (1-1,1*6)
                 = 6
\end{verbatim}
Tässä tapauksessa joudumme pitämään ainoastaan kaksi lukua muistissa koko laskennan ajan kun normaalissa rekursiossa joudumme pitämään kaikki rekursiokutsuketjun välivaiheet muistissa. Jos unohdetaan se, kuinka paljon luvun 3 ja 1000 000 esittäminen kuluttaa muistia, kertoma(3) ja kertoma(1000000) vaativat saman vakiomäärän muistia. 


\end{document}
