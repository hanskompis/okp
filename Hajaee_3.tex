\documentclass[11pt,oneside,a4paper]{article}
\usepackage{url}
\usepackage{hyperref}
\usepackage{lmodern}
\usepackage[pdftex]{graphicx}
\usepackage{textcomp}
\usepackage[utf8]{inputenc}
\usepackage[finnish]{babel}
\usepackage[T1]{fontenc}
\title{D- ja Erlang-kielten perustietotyypit ja laskennan kapselointi}
\author{Hansi Keijonen, Jari Koskinen, Eero Laine}

\begin{document}

\maketitle

\newpage

\section{Jarin D-osuus (kaikki D:stä aiheen ympärillä)}
\subsection{Alkeistyypit/perustyypit}
D-kielessä alkeistyyppejä \cite{ALE10} ovat: bool, byte, short, int, long, float, double ja real. Vastaavat etumerkittömät alkeistyypit ovat: ubyte, ushort, uint ja ulong. Näiden lisäksi on etumerkittömät merkkialkiot: char 8-bittiselle UTF8:lle ja wchar 16-bittiselle UTF16-merkille, jotka saavat alustuksen yhteydessä arvon 0xFF ja 0xFFFF. 32-bittinen UTF32-merkki voidaan tallettaa dchar alkeistyyppiin jonka määrittämätön alustusarvo on 0x0000FFFF. bool on aito totuusarvo ja voi saada arvon true tai false, alustuksen jälkeen määrittämätön arvo on false. byte on 8 bittiä pitkä tavu, short on 16, ja int 32 bittiä pitkä sana. Suurempia kokonaislukuja varten on 64 bitin pituinen long, joiden lisäksi kielessä on varaus cent tyypille, joka voi olla 128 bitin pituinen. Kaikki kokonaislukutyypit saavat muuttujan alustuksessa arvon 0. Reaaliluvuille varatut float, real ja double ovat alustuksen jälkeen nan, joka tarkoittaa Not A Number \cite{DLA13}. Kompleksiluvuille on alkeistyypit cfloat, cdouble ja creal, imaginääriluvuille ifloat, idouble ja ireal jotka saavat alustuksen yhteydessä arvon nan, mikäli arvoa ei ole määritetty.

Merkkijonoja varten on varattu string, joka koostuu taulukosta char-alkeistyyppejä. Merkkijonon tiettyyn alkioon voidaan viitata suoraan indeksillä:

\begin{verbatim}
string merkkijono = "Hei Lukija!";
for(int i=0; i<merkkijono.length; i++)
  write(merkkijono[i]);
\end{verbatim}
Koodi tulostaa:
\begin{verbatim}
Hei Lukija!
\end{verbatim}

Arvoalueet eivät varsinaisesti ole standardoituja, mutta kaikki D-kielen kääntäjät noudattavat samoja arvoalueita. Kielessä on käytössä vahva tyypitys ja tyyppitarkistukset tehdään staattisesti käännösaikana. D-kielessä on säännöt sille, miten eri alkeistyyppien välillä voidaan suorittaa laskentaa ja kuinka niitä voidaan muuntaa tyypistä toiseen. Muuttujan tyypin voi myös jättää kääntäjän päätettäväksi, määrittämällä muuttujan tyypiksi auto.

\begin{verbatim}
void main() {
  auto sana = "merkkijono";
  auto luku = 123456;
  writeln(sana);
  writeln(luku);
}
\end{verbatim}

D-kielessä etumerkittömät kokonaislukutyypit esiintyvät samanlaisina kuin C/C++ -kielissä. Tämä tekee alkeistyyppien käytöstä joustavampaa, verrattuna esimerkiksi Javaan, joka ei sisällä etumerkittömiä kokonaislukutyyppejä.

Alkeistyypeistä voidaan muodostaa taulukoita, joita on kahdenlaisia. Yksi taulukoista on yleinen perustapaus array, jollainen löytyy myös C/C++-kielistä. Taulukon alkioihin voidaan viitata indeksin avulla.
\begin{verbatim}
int[] lukuja;
lukuja[11] = 13;
\end{verbatim}

Toinen taulukkomuoto on avaimen ja arvon sisältävä pari. Taulukon alkio sisältää arvon ja siihen viittaava indeksi on esimerkiksi merkkijono: 
\begin{verbatim}
  int[string] kuukausi;
  
  kuukausi["tammikuu"] = 1;
  kuukausi["helmikuu"] = 2;
  
  writeln("Tammikuu=", kuukausi["tammikuu"]);
\end{verbatim}
Koodi tulostaa:
\begin{verbatim}
Tammikuu=1
\end{verbatim}

Alkeistyypeistä voidaan myös muodostaa tietueet struct tai union. Molemmat vastaavat C++-kielen vastaavia rakenteita ja ovat arvotyyppejä. Struct määritellään, ja sitä käytetään seuraavasti:

\begin{verbatim}
struct palkansaaja{
  int palkka;
  string titteli;
}

void main() {
  palkansaaja palkollinen;
  palkollinen.palkka = 4500;
  palkollinen.titteli = "ohjelmoija";

  write(palkollinen.titteli, " tienaa ");
  writeln(palkollinen.palkka, " kuukaudessa");
}
\end{verbatim}

Union eroaa structista siten, että yhden muuttujan arvon vaihtaminen muuttaa kaikkien saman arvotyypin muuttujien sisällön. Saman arvotyypin muuttujat sijoitetaan siis muistissa samalle muistialueelle. Seuraava esimerkki havainnolistaa tätä:
\begin{verbatim}
import std.stdio;

union tienaaja{
  int palkka;
  int tunnit;
  string titteli;
  string esimies;
}

void main() {
  tienaaja palkollinen;
  palkollinen.palkka = 6200;
  writeln(palkollinen.tunnit);
  writeln(palkollinen.titteli);
  palkollinen.esimies = "pomo";
  writeln(palkollinen.titteli);
}
\end{verbatim}
Ohjelma tulostaa seuraavasti:

\begin{verbatim}
6200
(null)
pomo
\end{verbatim}

%\begin{center}
%\begin{tabular}[t]{|l|l|c|c|}
%    \hline
%    Tyyppi & Kuvaus & Arvoalue & Alustus \\
%    \hline
%    bool & Totuusarvo & true, false & false \\
%    byte & 8 bitin tavu & -128 - 127 & 0 \\
%    ubyte & Etumerkitön 8 bitin tavu & 0 - 255 & 0 \\
%    short & 16 bitin sana & -32768 - 32767 & 0 \\
%    ushort & 16 bitin etumerkitön sana & 0 - 65535 & 0 \\
%    int & 32 bitin sana & -2147483647 - 2147483647 & 0 \\
%    uint & 32 bitin etumerkitön sana & 0 - 4294967295 & 0 \\
%    long & 64 bitin sana & -9223372036854775808 - 9223372036854775807 & 0 \\
%    \hline
%\end{tabular}
%\end{center}

... \\
... \\
... \\

Viittaukset menvät näin: \cite{ALE10}.

\section{Hansin osuus}
... \\
... \\
... \\

Viittaukset menvät näin: \cite{ERL99}.

\section{Eeron osuus}
... \\
... \\
... \\

Viittaukset menvät näin: \cite{HEB13}.

\bibliography{Lähteet}


\begin{thebibliography}{99}

\bibitem[ERL99]{ERL99} Erlang 4.7.3 Reference Manual, DRAFT (0.7), Jonas
Barklund, Robert Virding, 1999. 

\bibitem[DLA13]{DLA13} http://dlang.org, noudettu 6.2.2013.

\bibitem[HEB13]{HEB13} Learn You Some Erlang For Great Good, Fred Hébert, 2013.

\bibitem[ALE10]{ALE10} The D Programming Language, Andrei Alexandrescu, 2010.

\end{thebibliography}

\end{document}