\documentclass[11pt,oneside,a4paper]{article}
\usepackage{url}
\usepackage{hyperref}
\usepackage{lmodern}
\usepackage[pdftex]{graphicx}
\usepackage{textcomp}
\usepackage[utf8]{inputenc}
\usepackage[finnish]{babel}
\usepackage[T1]{fontenc}
\title{D- ja Erlang-kielten perustietotyypit ja laskennan kapselointi}
\author{Hansi Keijonen, Jari Koskinen, Eero Laine}

\begin{document}

\maketitle

\newpage

\section{Jarin D-osuus (kaikki D:stä aiheen ympärillä)}
\subsection{Alkeistyypit/perustyypit}
D-kielessä alkeistyyppejä \cite{ALE10} ovat: bool, byte, short, int, long,
float, double ja real. Vastaavat etumerkittömät alkeistyypit ovat: ubyte,
ushort, uint ja ulong. Näiden lisäksi on etumerkittömät merkkialkiot: char
8-bittiselle UTF8:lle ja wchar 16-bittiselle UTF16-merkille, jotka saavat
alustuksen yhteydessä arvon 0xFF ja 0xFFFF. 32-bittinen UTF32-merkki voidaan
tallettaa dchar alkeistyyppiin jonka määrittämätön alustusarvo on 0x0000FFFF.
bool on aito totuusarvo ja voi saada arvon true tai false, alustuksen jälkeen
määrittämätön arvo on false. byte on 8 bittiä pitkä tavu, short on 16, ja int 32
bittiä pitkä sana. Suurempia kokonaislukuja varten on 64 bitin pituinen long,
joiden lisäksi kielessä on varaus cent tyypille, joka voi olla 128 bitin
pituinen. Kaikki kokonaislukutyypit saavat muuttujan alustuksessa arvon 0.
Reaaliluvuille varatut float, real ja double ovat alustuksen jälkeen nan, joka
tarkoittaa Not A Number \cite{DLA13}. Kompleksiluvuille on alkeistyypit cfloat,
cdouble ja creal, 
imaginääriluvuille ifloat, idouble ja ireal jotka saavat alustuksen yhteydessä
arvon nan, mikäli arvoa ei ole määritetty.

Merkkijonoja varten on varattu string, joka koostuu taulukosta
char-alkeistyyppejä. Merkkijonon tiettyyn alkioon voidaan viitata suoraan
indeksillä:

\begin{verbatim}
string merkkijono = "Hei Lukija!";
for(int i=0; i<merkkijono.length; i++)
  write(merkkijono[i]);
\end{verbatim}
Koodi tulostaa:
\begin{verbatim}
Hei Lukija!
\end{verbatim}

Arvoalueet eivät varsinaisesti ole standardoituja, mutta kaikki D-kielen
kääntäjät noudattavat samoja arvoalueita. Kielessä on käytössä vahva tyypitys ja
tyyppitarkistukset tehdään staattisesti käännösaikana. D-kielessä on säännöt
sille, miten eri alkeistyyppien välillä voidaan suorittaa laskentaa ja kuinka
niitä voidaan muuntaa tyypistä toiseen. Muuttujan tyypin voi myös jättää
kääntäjän päätettäväksi, määrittämällä muuttujan tyypiksi auto.

\begin{verbatim}
void main() {
  auto sana = "merkkijono";
  auto luku = 123456;
  writeln(sana);
  writeln(luku);
}
\end{verbatim}

D-kielessä etumerkittömät kokonaislukutyypit esiintyvät samanlaisina kuin C/C++
-kielissä. Tämä tekee alkeistyyppien käytöstä joustavampaa, verrattuna
esimerkiksi Javaan, joka ei sisällä etumerkittömiä kokonaislukutyyppejä.

Alkeistyypeistä voidaan muodostaa taulukoita, joita on kahdenlaisia. Yksi
taulukoista on yleinen perustapaus array, jollainen löytyy myös C/C++-kielistä.
Taulukon alkioihin voidaan viitata indeksin avulla.
\begin{verbatim}
int[] lukuja;
lukuja[11] = 13;
\end{verbatim}

Toinen taulukkomuoto on avaimen ja arvon sisältävä pari. Taulukon alkio sisältää
arvon ja siihen viittaava indeksi on esimerkiksi merkkijono: 
\begin{verbatim}
  int[string] kuukausi;
  
  kuukausi["tammikuu"] = 1;
  kuukausi["helmikuu"] = 2;
  
  writeln("Tammikuu=", kuukausi["tammikuu"]);
\end{verbatim}
Koodi tulostaa:
\begin{verbatim}
Tammikuu=1
\end{verbatim}

Lisäksi taulukko voidaan jättää synaamiseksi, määrittämällä 
Alkeistyypeistä voidaan myös muodostaa tietueet struct tai union. Molemmat
vastaavat C++-kielen vastaavia rakenteita ja ovat arvotyyppejä \cite{DLA13}.
Struct määritellään, ja sitä käytetään seuraavasti:

\begin{verbatim}
struct palkansaaja{
  int palkka;
  string titteli;
}

void main() {
  palkansaaja palkollinen;
  palkollinen.palkka = 4500;
  palkollinen.titteli = "ohjelmoija";

  write(palkollinen.titteli, " tienaa ");
  writeln(palkollinen.palkka, " kuukaudessa");
}
\end{verbatim}

Union eroaa structista siten, että yhden muuttujan arvon vaihtaminen muuttaa
kaikkien saman arvotyypin muuttujien sisällön \cite{ALE10}. Saman arvotyypin
muuttujat sijoitetaan siis muistissa samalle muistialueelle. Seuraava esimerkki
havainnolistaa tätä:
\begin{verbatim}
import std.stdio;

union tienaaja{
  int palkka;
  int tunnit;
  string titteli;
  string esimies;
}

void main() {
  tienaaja palkollinen;
  palkollinen.palkka = 6200;
  writeln(palkollinen.tunnit);
  writeln(palkollinen.titteli);
  palkollinen.esimies = "pomo";
  writeln(palkollinen.titteli);
}
\end{verbatim}
Ohjelma tulostaa seuraavasti:

\begin{verbatim}
6200
(null)
pomo
\end{verbatim}

Lisäksi voidaan luoda dymaaninen taulukko, joka tarkoittaa sitä, että taulukkoon
voidaan myöhemmin sijoittaa taulukko. Esimerkki havainnolistaa tätä toimintaa:

\begin{verbatim}
  int[] c;
  int[4] d;
  d[2] = 10;
  c = d;
  writeln(c[2]);
\end{verbatim}
Ohjelma tulostaa luvun 10. \\
Taulukko voidaan myös osoittaa tiettyyn muistialueeseen. JATKA POINTER JNE. \\


D-kielen tarjoama struct ja union ovat useissa tilanteissa käyttökelpoisia,
varsinkin matalammalla tasolla, kuten esimerkiksi käyttöjärjestelmien
toteutuksessa. Koska ne ovat arvotyyppejä, niillä saadaan aikaan tehokkaita
rakenteita. D-kielen kehitykseen vaikuttaneista kielistä Java ei sisällä struct,
eikä union rakennetta. C\# sitä vastoin sisältää struct arvotyypin rakenteen
mutta ei unionia.

\subsection{Laskennan kapselointi}

Funktiot määritellään D-kielessä kuin C/C++-kielissä, funktiolle määritetään
ensin sen paluuarvon tyyppi, funktion nimi ja funktion mahdolliset parametrit.
Parametreille määritetään arvotyyppi. Funktion paluuarvon tyypin määrittäminen
voidaan myös jättää kääntäjän tehtäväksi, asettamalla arvotyypiksi auto.
Funktio, joka ei palauta mitään, asetetaan void tyyppiseksi, jolloin se vastaa
Pascal-kielen procedurea. Esimerkkejä erilaisista funktioista:

\begin{verbatim}
// palauttaa int-tyypin, parametreina 2 int-tyyppiä
int erotus(int a, int b) {
  return a - b;
}

// paluuarvon tyyppi jätetään kääntäjän päätettäväksi
auto summa(int a, int b) {
  return a - b;
}

// funktio, joka ei palauta mitään
void tulosta(string teksti) {
  writeln(teksti);
}
\end{verbatim}

Parametien välittämiseen voidaan käyttää arvo-, osoite- tai viiteparametreja.
Esimerkki osoite- ja viiteparametrien välittämisestä:
\begin{verbatim}
// osoiteparametrit
int summa(int *a, int *b) {
  return *a + *b;
}

// viiteparametrit
int summa(int &a, int &b) {
  return a + b;
}
\end{verbatim} 
D tarjoaa käyttöön myös aidon funktion, joka tarkoittaa funktiota, joka ei muuta
ohjelman tilaa tai käsittele globaaleja muuttujia \cite{DLA13}. Aito funktio ei
saa myöskään sisältää kutsua ei-aitoon funktioon. Tällainen funktio esitellään
lisäämällä funktion eteen määritys pure. Esimerkki aidosta funktiosta:

\begin{verbatim}
pure int tulo(int a, int b) {
  return a * b;
}
\end{verbatim}

D-kielessä funktioparametreille voidaan myös määrittää oletusarvot antamalla
parametrien esittelyn jälkeen niiden arvot.
\begin{verbatim}
int summa(int a=1, int b=1) {
  return a + b;
}
\end{verbatim}

\subsection{Virheenkäsittely}
Virheenkäsittely on D-kielessä toteutettu poikkeuksilla \cite{ALE10}, kuten
Javassa tai C\#-kielessäkin. D tukee try - catch - finally syntaktista
rakennetta virheekäsittelyssä. Poikkeus heitetään throw-lauseella, kuten
koodiesimerkissä on tehty.

\begin{verbatim}
int somethingElse() {
  throw new Exception("joku muu generoi virheen");
}

int something(int a) {
  return somethingElse();
}

void main() {
  try {
    something(10);
  }
  catch (Exception e) {
    writeln("Tapahtui virhe: ", e);
  }
  finally {
    writeln("Tulosta ainakin jotain...");
  }
}
\end{verbatim}

catch-lauseella määritetään mitä tehdään, jos poikkeus tapahtuu, ja
finally-lauseella määritetään tehtävät jotka suoritetaan lopuksi. Esimerkkinä
tällaisesta on tiedostonkäsittely; finally-lauseessa voidaan huolehtia siitä,
että tiedosto suljetaan asianmukaisesti.

\subsection{Rinnakkaisuuden tuki}
Rinnakaisuuteen D-kielessä on vahva tuki. D ei oletuksena käytä jaettua muistia
säikeiden ja prosessien kesken, vaan molempia ajetaan eristettynä
\cite{DLA13};\cite{ALE10}. Säikeiden ja prosessien välillä käytetään
viestinvälitystä joka tunnetaan nimellä \textit{message passing}. D sisältää
myös tuen perinteiselle \textit{kriittisen alueen} suojaukseen lukolla ja jaetun
muistin suojauksen \textit{mutexilla}. Suosituksena on kuitenkin käyttää
D-kielessä olevaa aktorimallia, jolloin säikeiden välillä hyödynnetään
viestijärjestelmää. receive-lause sisältää hahmontunnistuksen, jonka ansiosta
monimutkaisemmatkin viestit voidaan käsitellä niiden tyypin mukaisesti.
Esimerkissä viestejä lähetetään ja vastaanotetaan säikeiden välillä.

\begin{verbatim}
import std.concurrency, std.stdio, std.exception;
void main() {
  auto tid = spawn(&kirjoittaja);
  foreach(i; 0 .. 10) {
    // lähetä viesti
    tid.send("Hei maailma!");
    // lähetä oma Tid säikeelle
    tid.send(thisTid);
    receive(
      (string msg) { writeln("Main sai viestin: ", msg); }
    );
  }
}

void kirjoittaja() {
  for(;;) {
    receive(
      // jos saatiin string, kirjoita se ruudulle
      (string msg) { writeln("Thread sai viestin: ", msg); }, 
      // jos saatiin Tid, lähetä viesti
      (Tid tid) { tid.send("Hei!"); }
    );
  }
}

\end{verbatim}

Esimerkin ohjelma tulostaa ruudulle seuraavasti:

\begin{verbatim}
Thread sai viestin: Hei maailma!
Main sai viestin: Hei!
Thread sai viestin: Hei maailma!
Main sai viestin: Hei!
Thread sai viestin: Hei maailma!
Main sai viestin: Hei!
\end{verbatim}

D-kieli sisältää myös \textit{synchronized class} eli synkronoidun luokan, jota
voi käyttää rinnakkaisohjelmoinnissa suojaukseen.
\begin{verbatim}
synchronized class PankkiTili {
  private double _tili;

  void talleta(double summa) {
    _tili += summa;
  }

  void nosta(double summa) {
    enforce(_tili >= summa);
    _tili -= summa;
  }

  double saldo( ) {
    return _tili;
  }
}
\end{verbatim}
\subsection{Tyypillisiä ohjelmistoarkkitehtuureja}
D-kielellä voi toteuttaa arkkitehtuuriltaan hyvinkin erilaisia ohjelmistoja. Rinnakkaisuuden tuki mahdollistaa asiakas-palvelin-mallin, ja hajautettujen ohjelmistojen toteutuksen. Koska ohjelmoija voi valita toteutukseen D-kielen tarjoamista ohjelmointiparadigmoista imperatiivisen-, olio- tai funktionaalisen ohjelmoinnin, on D geneerisyydeltään käyttökelpoinen useimpiin arkkitehtuureihin. Kielellä tehtyjä laajempia toteutuksia ei ole kuitenkaan raportoitu julkisesti. Web-ohjelmointiin D ei sovellu yksin sellaisenaan, mutta esimerkiksi käyttöjärjestelmien ja sulautettujen järjestelmien ohjelmointi onnistuu hyvin.
\subsection{Valittujen ratkaisujen vertailua}
Kielessä on monipuolisia ja tehokkaita keinoja hallita rinnakkaisuutta.
Rinnakkaisuudessa tuettu viestinvälitys on uudemmissa kielissä myös yleinen.
Myös virheenkäsittely on toteutettu siten, että se on erittäin käyttökelpoinen.
try - catch - finally rakenne vaatii jonkin verran ohjelmoijalta paneutumista
asiaan, mutta on vaivan arvoinen. Funktioiden toteutus on perinteinen ja
noudattelee vahvasti suunnittelun mallina olleita C/C++/C\# kielten vastaavia. 
%\begin{center}
%\begin{tabular}[t]{|l|l|c|c|}
%    \hline
%    Tyyppi & Kuvaus & Arvoalue & Alustus \\
%    \hline
%    bool & Totuusarvo & true, false & false \\
%    byte & 8 bitin tavu & -128 - 127 & 0 \\
%    ubyte & Etumerkitön 8 bitin tavu & 0 - 255 & 0 \\
%    short & 16 bitin sana & -32768 - 32767 & 0 \\
%    ushort & 16 bitin etumerkitön sana & 0 - 65535 & 0 \\
%    int & 32 bitin sana & -2147483647 - 2147483647 & 0 \\
%    uint & 32 bitin etumerkitön sana & 0 - 4294967295 & 0 \\
%    long & 64 bitin sana & -9223372036854775808 - 9223372036854775807 & 0 \\
%    \hline
%\end{tabular}
%\end{center}

\section{Hansin osuus}
... \\
... \\
... \\

Viittaukset menvät näin: \cite{ERL99}.

\section{Erlangin perustietotyypit}

\subsection{Erlangin tukemat alkeistyypit (ja tietorakenteet?)}

Erlangissa on kahdentyyppisiä numeerisia literaaleja, kokonaislukuja ja
liukulukuja. Bit String
-tietotyyppiä käytetään varastoimaan tyypittämätöntä muistialuetta. Fun on
funktionaalinen objekti ja Port Identifier ilmaisee Erlang-portin. 

Atom-tietotyyppi on yksinkertaisesti nimetty vakio. Alla on esimerkkejä
atomeista.

\begin{verbatim}
heippa
ala_viiva
'Heippa'
'Hei hei'
\end{verbatim}
Tuple on kokoomatietotyyppi (compound data type), jolla on vakiomäärä termejä.
Termit ovat Erlangissa tiedon yksiköitä (pieces of data). Tuplen termejä
kutsutaan elementeiksi, ja elementtien määrä määrittää tuplen koon. Tämä
menettelytapa
muistuttaa taulukkorakennetta Java-kielessä.

Erlangissa on useita ennltamääriteltyjä funktioita tuplejen käsittelemiseksi.
Esimerkiksi:

\begin{verbatim}
Joukko = {paavi, 85}.
element(1,Joukko).
tuple_size(Joukko).
\end{verbatim}
Tässä tapauksessa element(1,P) viittaisi siis paavi-elementtiin ja Joukko-tuplen
koko olisi 2.

List on kokoomatietotyyppi, jolla on vaihteleva määrä termejä. Jokaista
List-tietotyypin termiä kutsutaan elementiksi, ja elementtien määrään viitataan
Listin pituutena Javan tapaan. Alla on esimerkki list-tietotyypistä.

\begin{verbatim}
Lista = [1,x,d,9] 
\end{verbatim}
String-listat eivät ole tietotyyppejä Erlangissa. Kaksoislainausmerkeissä oleva
merkkijono tarkoittaa sen sijaan samaa asiaa kuin lista merkkijonon sisältämistä
merkeistä. Esimerkiksi String \"heippa\" on sama asia kuin lista [\$h, \$e, \$i,
\$p, \$p, \$a]. 

Erlangissa ei ole Boolean-tietotyyppiä. Sen sijaan atomeja true ja false
käytetään ilmaisemaan totuusarvoja. 

\subsection{Erlangin tyypitys}

Dynaaminen tyypitys viittaa kieliin, joissa suurin osa tyyppien tarkastamisesta
tehdään vasta suoritusaikana käännösajan sijasta (Wikipedia). Erlangin
ensimmäisillä kehittäjillä oli tausta dynaamisesti tyypitettyjen kielten
parissa, joten dynaaminen tyypitys oli heille luonnollinen valinta. 

On usein esitetty, että staattisesti tyypitetyt kielet löytävät koodivirheet
paremmin, ja ovat siksi turvallisempia. Kuitenkin Erlangia paljon käyttäneellä
Ericsson-yhtiöllä on Erlangin turvallisuudesta huomattavaa näyttöä. Erlang on
rakennettu niin, että yhden komponentin epäonnistumisen ei tulisi vaikuttaa koko
järjestelmään. Ohjelmointivirheisiin, laitevikoihin ja joihinkin
tietoverkko-ongelmiin on varauduttu Erlangissa. Fred Hébertin mukaan Erlang
mahdollistaa odottamattomien virheiden käsittelyn, ohjelman jakamisen eri
solmuihin tietoverkossa ja suorituksen jatkamisen virheistä huolimatta.

Siinä missä useimmat kielet pyrkivät tekemään ohjelmista virheettömiä, Erlangin
strategia on hyväksyä virheiden tapahtuminen, ja varautua niihin. Tämän nojalla
Erlangin dynaaminen tyypitys ei ole este ohjelmien luotettavuudelle ja
turvallisuudelle.

Erlang on myös vahvasti tyypitetty kieli. Vahvasti tyypitetyt kielet rajoittavat
eri tietotyyppien yhdistelyä operaatioissa (wikipedia). Alla on esimerkki
operaatiosta, joka aiheuttaisi Erlangissa virheen ohjelman suorituksessa.

\begin{verbatim}
1 + "2".
** exception error: bad argument in an arithmetic expression
in operator  +/2
called as 1 + "2"
\end{verbatim}
Erlangissa on kuitenkin tapoja muuntaa tietoa tietotyypistä toiseen.

Viittaukset menvät näin: \cite{HEB13}.

\bibliography{Lähteet}


\begin{thebibliography}{99}

\bibitem[ERL99]{ERL99} Erlang 4.7.3 Reference Manual, DRAFT (0.7), Jonas
Barklund, Robert Virding, 1999. 

\bibitem[DLA13]{DLA13} http://dlang.org, noudettu 6.2.2013.

\bibitem[HEB13]{HEB13} Learn You Some Erlang For Great Good, Fred Hébert, 2013.

\bibitem[ALE10]{ALE10} The D Programming Language, Andrei Alexandrescu, 2010.

\end{thebibliography}

\end{document}