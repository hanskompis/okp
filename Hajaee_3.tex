\documentclass[11pt,oneside,a4paper]{article}
\usepackage{url}
\usepackage{hyperref}
\usepackage{lmodern}
\usepackage[pdftex]{graphicx}
\usepackage{textcomp}
\usepackage[utf8]{inputenc}
\usepackage[finnish]{babel}
\usepackage[T1]{fontenc}
\title{D- ja Erlang-kielten perustietotyypit ja laskennan kapselointi}
\author{Hansi Keijonen, Jari Koskinen, Eero Laine}

\begin{document}

\maketitle

\newpage

\section{Jarin D-osuus (kaikki D:stä aiheen ympärillä)}
... \\
... \\
... \\

Viittaukset menvät näin: \cite{ALE10}.

\section{Hansin osuus}
... \\
... \\
... \\

Viittaukset menvät näin: \cite{ERL99}.

\section{Eeron osuus}
... \\
... \\
... \\

Viittaukset menvät näin: \cite{HEB13}.

\bibliography{Lähteet}


\begin{thebibliography}{99}

\bibitem[ERL99]{ERL99} Erlang 4.7.3 Reference Manual, DRAFT (0.7), Jonas
Barklund, Robert Virding, 1999. 

\bibitem[DLA13]{DLA13} http://dlang.org, noudettu 6.2.2013.

\bibitem[HEB13]{HEB13} Learn You Some Erlang For Great Good, Fred Hébert, 2013.

\bibitem[ALE10]{ALE10} The D Programming Language, Andrei Alexandrescu, 2010.

\end{thebibliography}

\end{document}