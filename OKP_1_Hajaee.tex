\documentclass[11pt,oneside,a4paper]{article}
%\usepackage{ucs}
\usepackage{url}
\usepackage{hyperref}
\usepackage[pdftex]{graphicx}
\usepackage{textcomp}
\usepackage[utf8]{inputenc}
\usepackage[finnish]{babel}
\usepackage[T1]{fontenc}
\author{Jari Koskinen, Hansi Keijonen, Eero Laine}
\title{Ohjelmointikielten periaatteet 2013}




\begin{document}
% kirjoita nimiö
\maketitle

\pagebreak

\tableofcontents

\pagebreak

\section{D kielen esittely}
\subsection{Yleistä}
D kielessä on ominaisuuksia plaaplaa...
\begin{itemize}
\item Automaattinen roskienkeruu
\item Vahva tyypitys
\item Käännös natiivikoodiksi
\item Rinnakkaisuuden tuki
\end{itemize}
Lisää kielestä plaaplaa...

\section{Perusteet}
\subsection{Tietotyypit}
D on vahvasti tyypitetty kieli...
Seuraavat tietotyypit ovat tuettuna:
\begin{itemize}
\item Int
\item Double
\item Char
\item ...
\end{itemize}
\subsection{Rakenne}
Näin tulostetaan Hello World!
\begin{verbatim}
import std.stdio;

void main() {
 writeln("Hello World!");
}
\end{verbatim}
C-kielestä tuttu tapa on mahdollinen...

\begin{verbatim}
import std.stdio;

void main() {
  for(int i=0; i<10; i++) {
    writeln("Rivi: ", i);
  }
}
\end{verbatim}
Iteraatio voidaan tehdä myös seuraavasti:
\begin{verbatim}
import std.stdio;

void main() {
  foreach(i; 0 .. 10) {
    writeln("Rivi: ", i);
  }
}
\end{verbatim}

ja tuloste näyttää kuvan \ref{konsoli1} mukaiselta.
\begin{figure}[tbh]
%\begin{figure}[tbh] t= top, b = bottom, h=here
\begin{center}
\includegraphics[width=1.0\textwidth]{konsoli1.jpg}
%\rotatebox{90}{\includegraphics[scale=.75]{esimerkki.pdf}}
\caption{Tuloste konsolilla}
\label{konsoli1}
\end{center}
\end{figure}

\section{Yhtenveto}
Yhteenvetona todettakoon, että D on ...
\subsection{Rinnakkaisuus}
Rinnakkaisuus on toteutettu ...
\end{document}